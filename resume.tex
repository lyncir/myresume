\documentclass[11pt,letterpaper]{article}

%-----------------------------------------------------------
\usepackage{CJK}
\usepackage{array}
\usepackage{url}

\usepackage[empty]{fullpage}
\textheight=9.0in
\raggedbottom
\raggedright
\setlength{\tabcolsep}{0in}

% Adjust margins
\addtolength{\oddsidemargin}{-0.375in}
\addtolength{\evensidemargin}{0.375in}
\addtolength{\textwidth}{0.5in}
\addtolength{\topmargin}{-.375in}
\addtolength{\textheight}{0.75in}

\usepackage{xcolor}
\definecolor{bl3}{RGB}{32,57,109}

\usepackage{titlesec}
\titleformat{\section}{\color{bl3}\large\scshape\bfseries\raggedright}{}{0em}{}[\titlerule]
\titlespacing{\section}{0pt}{3pt}{3pt}

%-----------------------------------------------------------
%Custom commands
\newcommand{\twocol}[2]{
\begin{tabular}{p{1.4cm}l}
	\textit{#1} & #2\\
\end{tabular}}
\newcommand{\resitem}[1]{\item #1 \vspace{-2pt}}
\newcommand{\ressubheading}[4]{
\begin{tabular*}{6.5in}{l@{\extracolsep{\fill}}r}
	\textbf{#1} & \url{#2} \\
	\textit{#3} & \textit{#4} \\
\end{tabular*}\vspace{-6pt}}
%-----------------------------------------------------------


\begin{document}
\begin{CJK}{UTF8}{gbsn}

%-----------------Info-------------------
\begin{tabular}{p{1.2cm}l}
	姓名: & 陈昌明 \\
	电话: & 18500194491 \\
	邮箱: & lyncir@gmail.com \\
	博客: & www.lyncir.com \\
\end{tabular}
\\
\vspace{0.1in}

%----------------Education---------------
\section{教育}  
	\begin{itemize}
	 	\item    
			\ressubheading{福州大学 - 工程技术学院}{http://etc.fzu.edu.cn}{网络系统管理}{2008.07 -- 2011.07}
	\end{itemize}

%----------------Experience--------------
\section {经验}   
	\begin{itemize}
		\item 
			\ressubheading{北京飞流科技有限公司}{http://www.feiliu.com}{Linux系统运维工程师}{2012.11 -- Present}
				{ \footnotesize
				\begin{itemize}
						\resitem{负责对公司Apps的数据日志文件进行分析并入库,以提供数据报表}
						\resitem{负责一款手机游戏项目服务器的基础运维,如:python环境,Apache,Redis,数据备份还原等}
						\resitem{负责部分项目的WEB前端及后端编写}
						\resitem{负责部分爬虫应用的编写}
				\end{itemize}
			 	}     
		\item 
			\ressubheading{盘丝无限科技有限公司}{http://www.pansi.me}{Linux系统运维工程师}{2011.07 -- 2012.11}
				{ \footnotesize
				\begin{itemize}
						\resitem{负责Linux平台服务器的运维和管理:包括配置,监控,系统性能优化,网络安全及备份等}
						\resitem{负责Linux服务器架构设计,规划和实施,以及负载均衡和故障切换}
						\resitem{负责公司内部服务基础建设,如:LDAP,Git,VsFTP,Hudson,Trac,Testlink,Bugzilla,OpenVPN,Confluence,Jira,Crowd等}
						\resitem{负责Linux下常用脚本编写,对其他人员提供linux系统使用支持}
				\end{itemize}
			 	}     
	\end{itemize}

%----------------Projects------------------
\section{项目}  
       \begin{itemize}
		   \item{\textbf{飞流应用监控平台}}\vspace{-6pt}
                       {\footnotesize
                       \begin{itemize}
							   \resitem{主要原理是发送虚拟的请求报文,检测App是否正常运行}
                               \resitem{我负责平台的WEB编写,使用Flask+Bootstrap模板}
                       \end{itemize}
                       } 
		   \item{\textbf{积分墙API}}\vspace{-6pt}
                       {\footnotesize
                       \begin{itemize}
							   \resitem{主要使用Flask+Redis做队列服务}
                               \resitem{这块并发要求比较高,需要耐心的调整.第二就是命中率问题.Redis即做缓存又做队列服务}
                       \end{itemize}
                       } 
               \item{\textbf{盘丝服务器架构和部署}}\vspace{-6pt}
                       {\footnotesize
                       \begin{itemize}
                               \resitem{前端使用HAProxy做负载均衡}
                               \resitem{后端为Openfire和Tomcat应用软件,数据库为MySQL}
                               \resitem{其中MySQL之间互为主从进行数据同步}
                       \end{itemize}
                       } 
               \item{\textbf{盘丝监控服务器部署}}\vspace{-6pt}
                       {\footnotesize
                       \begin{itemize}
                               \resitem{使用Cacti,对服务器资源进行监控(如:cpu,memory,network,haproxy,apache,tomcat,mysql等)}
                               \resitem{使用Nagios,对服务器的服务进行监控(如:HTTP Port,MySQL Replication等),并进行警报通知}
                       \end{itemize}
                       }
		\end{itemize}

%----------------Skills------ ---------  
\section{技能}
	\begin{itemize}
		\item
			\textbf{Linux}
			{ \footnotesize
			\begin{enumerate}
				\resitem{Linux忠实粉丝,使用已有4年.熟悉的系统有gentoo和CentOS}
			\end{enumerate}
			}
		\item
			\textbf{Python}
			{ \footnotesize
			\begin{enumerate}
				\resitem{主要使用Flask Web框架}
				\resitem{习惯使用BeautifulSoup4编写一些小爬虫}
				\resitem{喜欢的库有: requests, dateutil, trondb, pymysql}
			\end{enumerate}
			}
	\end{itemize}

\end{CJK} 
\end{document}
